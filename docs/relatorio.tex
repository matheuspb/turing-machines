\documentclass[12pt]{article}

\usepackage[a4paper, margin=2.5cm]{geometry}
\usepackage[brazilian]{babel}
\usepackage[utf8]{inputenc}
\usepackage{lmodern}
\usepackage{indentfirst}
\usepackage{amssymb}
\usepackage{amsmath}

\title{Máquinas de Turing com JFLAP}
\date{\today}
\author{Filipe Borba \and Matheus Bittencourt}

\begin{document}

\maketitle

\section{Máquinas de fita única}

\subsection*{(a) \(L = \{a^i b^j c^k \mid i,j,k \in \mathbb{N} \text{\rm{ e }}
	i \times j = k\}\)}

Para cada `a', trocamos ele por `A', depois para cada `b' testamos se temos um
`c' correspondente, fazendo isso para todos os `a's. Ao final da marcação de
letras C, as letras B são desmarcadas e a próxima letra A é marcada, e assim
por diante. Se a multiplicação não apresentar seu resultado correto, lembrando
que todas as letras precisam aparecer pelo menos uma vez, a máquina rejeitará a
entrada.

\subsection*{(b) \(L = \{\#x_1\#x_2\#\ldots\#x_n \mid x_i \in \{0,1\}^*
	\text{\rm{ e }} x_i \ne x_j \text{\rm{ para cada }} i \ne j\}\)}

A máquina compara $x_i$ e $x_j$, $\forall i\neq j$, exceto pela palavra vazia,
garantida no início do procedimento. Após esta garantia, um $x_i$ é fixado e
comparado com $x_{i+1}, x_{i+2}, \cdots, x_n$. Não havendo subpalavras iguais
nesta iteração a próxima subpalavra, $x_{i+1}$, é fixada e comparada com os
elementos posteriores, separados pelo \#, até que este não seja mais encontrado
na entrada, significando o fim da mesma.

\section{Máquinas com múltiplas fitas}

\subsection*{(a) \(L = \{ww \mid w \in \{0,1\}^*\}\)}

Para computar essa linguagem usando 2 fitas, movemos o primeiro símbolo para a
segunda fita, e comparamos as duas fitas, se elas possuirem os mesmos símbolos,
aceitamos a \textit{string}, senão retrocedemos a primeira fita até o novo
primeiro símbolo, copiamos esse símbolo para a segunda fita, e repetimos o
processo de comparação das duas fitas. Repetimos esses passos até tenhamos duas
\textit{strings} semelhantes nas duas fitas, aceitando a \textit{string}, ou
até que a primeira fita esteja vazia, rejeitando a \textit{string}.

\subsection*{(b) \(L = \{a^n b^n c^n \mid n \geq 0\}\)}

Usando 2 fitas, copiamos todos os `a's para a segunda fita, retrocedemos o
cabeçote da segunda fita até o início dela. Depois, para cada `b' na primeira
fita checamos se existe um `a' na segunda fita. Retrocedemos o cabeçote da
segunda fita, e repetimos o processo para os `c's. Se chegarmos ao final da
primeira fita aceitamos a \textit{string}.

\section{Máquinas em bloco}

\subsection*{(a) \(L = \{{0^2}^n \mid n \geq 0\}\)}

Se tirarmos o primeiro zero de uma \textit{string} $w$ tal que $|w| = 2^n$,
então sobrará um número ímpar de zeros, e sempre que dividirmos esse número de
zeros por 2 (truncando o resultado da divisão) obteremos um número ímpar
também. Se dividirmos $n - 1$ vezes por 2, o resultado será sempre 1. Prova:

\begin{align}
	\lfloor\frac{2^n - 1}{2^{n-1}}\rfloor &= 1 \\
	\lfloor\frac{2^n}{2^{n-1}} - \frac{1}{2^{n-1}}\rfloor &= 1 \\
	\lfloor2 - \frac{1}{2^{n-1}}\rfloor &= 1 \label{final}
\end{align}

sendo $n \geq 1$, então $0 < \frac{1}{2^{n - 1}} \leq 1$ portanto \ref{final}
está correto.

Tendo isso em mente, para computar essa linguagem, apagamos o primeiro `0' da
fita , trocamos o próximo `0' da fita por `x' e pulamos um `0', marcando apenas
metade dos zeros. Se nessa passada pela fita marcando os `0's tivermos um
número par de `0's, rejeitamos a \textit{string}, senão continuamos o processo
até que toda a \textit{string} seja marcada por `x's.

\subsection*{(b) \rm Um somador binário}

Assumindo que a entrada dessa máquina seja algo como ``1001+1010'', para
computar a soma, caminhamos até o final da fita, apagamos o número que está lá,
caminhamos até o bit menos significativo do primeiro número, que ainda não foi
marcado, e marcamos o valor já adicionado, que se for 0, marcamos `a', e 1
marcamos `b'. Se a soma desses dois bits gerar um \textit{carry} continuamos
somando nos números adjacentes, porém sem marcá-los, até que não tenha mais
esse \textit{carry}. Repetimos esse processo até que o segundo número esteja
vazio e depois trocamos todos os `a's e `b's por `0's e `1's, respectivamente.

\end{document}
